\documentclass[aspectratio = 169, 13pt]{beamer}

% xcolor and define colors -------------------------
\usepackage{xcolor}

% https://www.viget.com/articles/color-contrast/
\definecolor{navy}{HTML}{567293}
\definecolor{purple}{HTML}{695693}
\definecolor{ruby}{HTML}{9a2515}
\definecolor{alice}{HTML}{107895}
\definecolor{daisy}{HTML}{EBC944}
\definecolor{coral}{HTML}{F26D21}
\definecolor{kelly}{HTML}{829356}
\definecolor{cranberry}{HTML}{E64173}
\definecolor{jet}{HTML}{131516}
\definecolor{asher}{HTML}{555F61}
\definecolor{slate}{HTML}{314F4F}

\newcommand\navy[1]{{\color{navy}#1}}
\newcommand\purple[1]{{\color{purple}#1}}
\newcommand\kelly[1]{{\color{kelly}#1}}
\newcommand\ruby[1]{{\color{ruby}#1}}
\newcommand\alice[1]{{\color{alice}#1}}
\newcommand\daisy[1]{{\color{daisy}#1}}
\newcommand\coral[1]{{\color{coral}#1}}
\newcommand\cranberry[1]{{\color{cranberry}#1}}
\newcommand\slate[1]{{\color{slate}#1}}
\newcommand\jet[1]{{\color{jet}#1}}
\newcommand\asher[1]{{\color{asher}#1}}

\newcommand\bgNavy[1]{{\colorbox{navy!80!white}{#1}}}
\newcommand\bgPurple[1]{{\colorbox{purple!80!white}{#1}}}
\newcommand\bgKelly[1]{{\colorbox{kelly!80!white}{#1}}}
\newcommand\bgRuby[1]{{\colorbox{ruby!80!white}{#1}}}
\newcommand\bgAlice[1]{{\colorbox{alice!80!white}{#1}}}
\newcommand\bgDaisy[1]{{\colorbox{daisy!80!white}{#1}}}
\newcommand\bgCoral[1]{{\colorbox{coral!80!white}{#1}}}
\newcommand\bgCranberry[1]{{\colorbox{cranberry!80!white}{#1}}}

% Mixtape Sessions
\definecolor{picton-blue}{HTML}{00b7ff}
\definecolor{violet-red}{HTML}{ff3881}
\definecolor{sun}{HTML}{ffaf18}
\definecolor{electric-violet}{HTML}{871EFF}

\newcommand\pictonBlue[1]{{\color{picton-blue}#1}}
\newcommand\sun[1]{{\color{sun}#1}}
\newcommand\electricViolet[1]{{\color{electric-violet}#1}}
\newcommand\violetRed[1]{{\color{violet-red}#1}}

\newcommand\bgPictonBlue[1]{{\colorbox{picton-blue!20!white}{#1}}}
\newcommand\bgSun[1]{{\colorbox{sun!20!white}{#1}}}
\newcommand\bgElectricViolet[1]{{\colorbox{electric-violet!20!white}{#1}}}
\newcommand\bgVioletRed[1]{{\colorbox{violet-red!20!white}{#1}}}

\def\code#1{\texttt{#1}}


% Main theme colors
\definecolor{accent}{HTML}{00b7ff}
\definecolor{accent2}{HTML}{871EFF}
\definecolor{gray100}{HTML}{f3f4f6}
\definecolor{gray800}{HTML}{1F292D}


% Beamer Options -------------------------------------

% Background
\setbeamercolor{background canvas}{bg = white}

% Change text margins
\setbeamersize{text margin left = 15pt, text margin right = 15pt} 

% \alert
\setbeamercolor{alerted text}{fg = accent2}

% Frame title
\setbeamercolor{frametitle}{bg = white, fg = jet}
\setbeamercolor{framesubtitle}{bg = white, fg = accent}
\setbeamerfont{framesubtitle}{size = \small, shape = \itshape}

% Block
\setbeamercolor{block title}{fg = white, bg = accent2}
\setbeamercolor{block body}{fg = gray800, bg = gray100}

% Title page
\setbeamercolor{title}{fg = gray800}
\setbeamercolor{subtitle}{fg = accent}

%% Custom \maketitle and \titlepage
\setbeamertemplate{title page}
{
    %\begin{centering}
        \vspace{20mm}
        {\Large \usebeamerfont{title}\usebeamercolor[fg]{title}\inserttitle}\\
        {\large \itshape \usebeamerfont{subtitle}\usebeamercolor[fg]{subtitle}\insertsubtitle}\\ \vspace{10mm}
        {\insertauthor}\\
        {\color{asher}\small{\insertdate}}\\
    %\end{centering}
}

% Table of Contents
\setbeamercolor{section in toc}{fg = accent!70!jet}
\setbeamercolor{subsection in toc}{fg = jet}

% Button 
\setbeamercolor{button}{bg = accent}

% Footnotes in Navigation Symbols
\setbeamertemplate{navigation symbols}{}

\usepackage{appendixnumberbeamer}
\setbeamercolor{page number in head/foot}{fg=alice}
\setbeamertemplate{footline}[frame number]


% Table and Figure captions
\setbeamercolor{caption}{fg=jet!70!white}
\setbeamercolor{caption name}{fg=jet}
\setbeamerfont{caption name}{shape = \itshape}

% Bullet points

%% Jon's itemize
\setbeamersize{text margin left=1em,text margin right=1em} 
\newenvironment{wideitemize}{\itemize\addtolength{\itemsep}{10pt}}{\enditemize}
\newenvironment{wideitemizeshort}{\itemize}{\enditemize}

%% Fix left-margins
\settowidth{\leftmargini}{\usebeamertemplate{itemize item}}
\addtolength{\leftmargini}{\labelsep}

%% enumerate item color
\setbeamercolor{enumerate item}{fg = accent}
\setbeamerfont{enumerate item}{size = \small}
\setbeamertemplate{enumerate item}{\insertenumlabel.}

\setbeamercolor{enumerate subitem}{fg = accent!60!white}
\setbeamerfont{enumerate subitem}{size = \small}
\setbeamertemplate{enumerate subitem}{\insertenumlabel.}

%% itemize
\setbeamercolor{itemize item}{fg = accent!70!white}
\setbeamerfont{itemize item}{size = \small}
\setbeamertemplate{itemize item}[circle]

%% right arrow for subitems
\setbeamercolor{itemize subitem}{fg = accent!60!white}
\setbeamerfont{itemize subitem}{size = \small}
\setbeamertemplate{itemize subitem}{$\rightarrow$}

\setbeamertemplate{itemize subsubitem}[square]
\setbeamercolor{itemize subsubitem}{fg = jet}
\setbeamerfont{itemize subsubitem}{size = \small}







% Links ----------------------------------------------

\usepackage{hyperref}
\hypersetup{
  colorlinks = true,
  linkcolor = accent2,
  filecolor = accent2,
  urlcolor = accent2,
  citecolor = accent2,
}


% Line spacing --------------------------------------
\usepackage{setspace}
\setstretch{1.2}


% \begin{columns} -----------------------------------
\usepackage{multicol}


% Fonts ---------------------------------------------
% Beamer Option to use custom fonts
\usefonttheme{professionalfonts}

% \usepackage[utopia, smallerops, varg]{newtxmath}
% \usepackage{utopia}
\usepackage[sfdefault,light]{roboto}

% Small adjustments to text kerning
\usepackage{microtype}



% Remove annoying over-full box warnings -----------
\vfuzz2pt 
\hfuzz2pt


% Table of Contents with Sections
\setbeamerfont{myTOC}{series=\bfseries, size=\Large}
\AtBeginSection[]{
        \frame{
            \frametitle{Roadmap}
            \tableofcontents[current]   
        }
    }


% Tables -------------------------------------------
% Tables too big
% \begin{adjustbox}{width = 1.2\textwidth, center}
\usepackage{adjustbox}
\usepackage{array}
\usepackage{threeparttable, booktabs, adjustbox}
    
% Fix \input with tables
% \input fails when \\ is at end of external .tex file
\makeatletter
\let\input\@@input
\makeatother

% Tables too narrow
% \begin{tabularx}{\linewidth}{cols}
% col-types: X - center, L - left, R -right
% Relative scale: >{\hsize=.8\hsize}X/L/R
\usepackage{tabularx}
\newcolumntype{L}{>{\raggedright\arraybackslash}X}
\newcolumntype{R}{>{\raggedleft\arraybackslash}X}
\newcolumntype{C}{>{\centering\arraybackslash}X}

% Figures

% \imageframe{img_name} -----------------------------
% from https://github.com/mattjetwell/cousteau
\newcommand{\imageframe}[1]{%
    \begin{frame}[plain]
        \begin{tikzpicture}[remember picture, overlay]
            \node[at = (current page.center), xshift = 0cm] (cover) {%
                \includegraphics[keepaspectratio, width=\paperwidth, height=\paperheight]{#1}
            };
        \end{tikzpicture}
    \end{frame}%
}

% subfigures
\usepackage{subfigure}


% Highlight slide -----------------------------------
% \begin{transitionframe} Text \end{transitionframe}
% from paulgp's beamer tips
\newenvironment{transitionframe}{
    \setbeamercolor{background canvas}{bg=accent!40!black}
    \begin{frame}\color{accent!10!white}\LARGE\centering
}{
    \end{frame}
}


% Table Highlighting --------------------------------
% Create top-left and bottom-right markets in tabular cells with a unique matching id and these commands will outline those cells
\usepackage[beamer,customcolors]{hf-tikz}
\usetikzlibrary{calc}
\usetikzlibrary{fit,shapes.misc}
\usetikzlibrary{decorations.pathreplacing}


% To set the hypothesis highlighting boxes red.
\newcommand\marktopleft[1]{%
    \tikz[overlay,remember picture] 
        \node (marker-#1-a) at (0,1.5ex) {};%
}
\newcommand\markbottomright[1]{%
    \tikz[overlay,remember picture] 
        \node (marker-#1-b) at (0,0) {};%
    \tikz[accent!80!jet, ultra thick, overlay, remember picture, inner sep=4pt]
        \node[draw, rectangle, fit=(marker-#1-a.center) (marker-#1-b.center)] {};%
}



% References ----------------------------------------

%% Bibliography Font, roughly matching aea
\setbeamerfont{bibliography item}{size = \footnotesize}
\setbeamerfont{bibliography entry author}{size = \footnotesize, series = \bfseries}
\setbeamerfont{bibliography entry title}{size = \footnotesize}
\setbeamerfont{bibliography entry location}{size = \footnotesize, shape = \itshape}
\setbeamerfont{bibliography entry note}{size = \footnotesize}

\setbeamercolor{bibliography item}{fg = jet}
\setbeamercolor{bibliography entry author}{fg = accent!60!jet}
\setbeamercolor{bibliography entry title}{fg = jet}
\setbeamercolor{bibliography entry location}{fg = jet}
\setbeamercolor{bibliography entry note}{fg = jet}

%% Remove bibliography symbol in slides
\setbeamertemplate{bibliography item}{}


% Citations ----------------------------------------
\usepackage{natbib}
%\bibliographystyle{apalike}
\bibliographystyle{aer}

\usepackage{bibunits}
\defaultbibliography{Bibliography.bib}  
%\defaultbibliographystyle{apalike}
\defaultbibliographystyle{aer}

\newcommand{\backupbegin}{
  \newcounter{finalframe}
  \setcounter{finalframe}{\value{framenumber}}
}
\newcommand{\backupend}{
  \setcounter{framenumber}{\value{finalframe}}
}

% Probability
\renewcommand{\P}{\mathop{}\!\textnormal{P}}
\newcommand{\E}{\mathop{}\!\textnormal{E}}
\newcommand{\sE}{\mathop{}\!\mathbb{E}}
\newcommand{\Cov}{\mathop{}\!\textnormal{Cov}}
\newcommand{\sCov}{\mathop{}\!\mathbb{C}\textnormal{ov}}
\newcommand{\Var}{\mathop{}\!\textnormal{Var}}
\newcommand{\sVar}{\mathop{}\!\mathbb{V}\textnormal{ar}}
\newcommand{\N}{\mathcal{N}}
\newcommand{\sN}{\N(0,1)}

\newcommand{\bracks}[1]{\left[#1\right]}
\newcommand{\expe}[1]{\mathbb{E}\bracks{#1}}
\newcommand{\expestar}[1]{\mathbb{E}^*\bracks{#1}}
\newcommand{\expesub}[2]{\mathbb{E}_{#1}\bracks{#2}}

\newcommand{\var}[1]{\mathbb{V}\text{ar}\bracks{#1}}
\newcommand{\parens}[1]{\left(#1\right)}
  \newcommand{\squared}[1]{\parens{#1}^2}
  \newcommand{\tothepow}[2]{\parens{#1}^{#2}}
  \newcommand{\prob}[1]{\mathbb{P}\parens{#1}}
  \newcommand{\probsub}[2]{\mathbb{P}_{#1}\parens{#2}}

% Convergence
\newcommand{\Oh}{\mathop{}\!\mathcal{O}}
\newcommand{\oh}{\mathop{}\!{o}}
\newcommand{\convd}{\stackrel{d}{\longrightarrow}}
\newcommand{\convp}{\stackrel{p}{\longrightarrow}}
\newcommand{\conv}{\stackrel{}{\longrightarrow}}

% Basic matrices and vectors
\newcommand{\0}{\mathbf{0}}
\newcommand{\1}{\mathbf{1}}
\newcommand{\I}{\mathbb{I}}
\renewcommand{\O}{\mathbb{O}}

% Basic maths
\newcommand{\R}{\mathbb{R}}
\DeclareMathOperator*{\argmin}{arg\,min}
\DeclareMathOperator*{\argmax}{arg\,max}
\newcommand{\Ind}{\mathbbm{1}}  % Indicator
\newcommand{\reals}{\mathbb{R}}
% Estimation specifics
  \newcommand{\Ih}{\widehat{I}}
  \newcommand{\Is}{\mathcal{I}}
  
% Independence
  \newcommand{\orth}{\perp}
  \renewcommand{\given}{\: | \:}
  
% Prediction
\newcommand{\X}{\mathcal{X}}  % Covariate space
\newcommand{\W}{\mathcal{W}}  % Transformed covariate space
\newcommand{\Y}{\mathcal{Y}}  % Outcome space
\newcommand{\err}{\varepsilon}  % Outcome error
\newcommand{\g}{g}  % Truth (conditional expectation/optimal predictor)
\newcommand{\F}{\mathcal{F}}  % Function class
\renewcommand{\S}{\ensuremath{S}}  % Sample
\newcommand{\T}{\ensuremath{T}}  % Training sample
\renewcommand{\H}{\ensuremath{H}}  % Hold-out sample
\newcommand{\f}{f}  % Prediction function
\newcommand{\fh}{\ensuremath{\widehat{f}}}  % Empirical prediction
\newcommand{\yhat}{\ensuremath{\widehat{y}}}	
\newcommand{\betahat}{\ensuremath{\widehat{\beta}}}
\newcommand{\betatilde}{\ensuremath{\tilde{\beta}}}
\newcommand{\deltatilde}{\ensuremath{\tilde{\delta}}}
\newcommand{\xtilde}{\ensuremath{\tilde{x}}}
\newcommand{\Xtilde}{\ensuremath{\tilde{X}}}

\newcommand{\al}{\ensuremath{\mathcal{L}}}
\newcommand{\OLS}{\textnormal{OLS}}  % Sendhil does not know how to spell this
\newcommand{\lh}{\widehat{\lambda}}
  \newcommand{\Z}{\mathcal{Z}}
  \renewcommand{\th}{\widehat{\theta}}
\newcommand{\Sigmahat}{\ensuremath{\widehat{\Sigma}}}
\newcommand{\sigmahat}{\ensuremath{\widehat{\sigma}}}

%Distributions
\newcommand{\logistic}[2]{\mathrm{Logistic}\parens{#1,\,#2}}
\newcommand{\bernoulli}[1]{\mathrm{Bernoulli}\parens{#1}}
\newcommand{\betanot}[2]{\mathrm{Beta}\parens{#1,\,#2}}
\newcommand{\stdbetanot}{\betanot{\alpha}{\beta}}
\newcommand{\multnormnot}[3]{\mathcal{N}_{#1}\parens{#2,\,#3}}
\newcommand{\normnot}[2]{\mathcal{N}\parens{#1,\,#2}}
\newcommand{\classicnormnot}{\normnot{\mu}{\sigsq}}
\newcommand{\stdnormnot}{\normnot{0}{1}}
\newcommand{\uniform}[2]{\mathrm{U}\parens{#1,\,#2}}
\newcommand{\stduniform}{\uniform{0}{1}}
\newcommand{\exponential}[1]{\mathrm{Exp}\parens{#1}}
\newcommand{\stdexponential}{\mathrm{Exp}\parens{1}}
\newcommand{\gammadist}[2]{\mathrm{Gamma}\parens{#1, #2}}
\newcommand{\poisson}[1]{\mathrm{Poisson}\parens{#1}}
\newcommand{\geometric}[1]{\mathrm{Geometric}\parens{#1}}
\newcommand{\binomial}[2]{\mathrm{Binomial}\parens{#1,\,#2}}
\newcommand{\rayleigh}[1]{\mathrm{Rayleigh}\parens{#1}}
\newcommand{\multinomial}[2]{\mathrm{Multinomial}\parens{#1,\,#2}}
\newcommand{\gammanot}[2]{\mathrm{Gamma}\parens{#1,\,#2}}
\newcommand{\cauchynot}[2]{\text{Cauchy}\parens{#1,\,#2}}
\newcommand{\invchisqnot}[1]{\text{Inv}\chisq{#1}}
\newcommand{\invscaledchisqnot}[2]{\text{ScaledInv}\ncchisq{#1}{#2}}
\newcommand{\invgammanot}[2]{\text{InvGamma}\parens{#1,\,#2}}
\newcommand{\chisq}[1]{\chi^2_{#1}}
\newcommand{\ncchisq}[2]{\chi^2_{#1}\parens{#2}}
\newcommand{\ncF}[3]{F_{#1,#2}\parens{#3}}
\newcommand{\Fnot}[2]{F\parens{#1, #2}}



%shortcuts for Linear Algebra stuff (i.e. vectors and matrices)
\newcommand{\twovec}[2]{\parens{\begin{array}{c} #1 \\ #2 \end{array}}}
\newcommand{\threevec}[3]{\parens{\begin{array}{c} #1 \\ #2 \\ #3 \end{array}}}
\newcommand{\fivevec}[5]{\parens{\begin{array}{c} #1 \\ #2 \\ #3 \\ #4 \\ #5 \end{array}}}
\newcommand{\twobytwomat}[4]{\parens{\begin{array}{cc} #1 & #2 \\ #3 & #4 \end{array}}}
\newcommand{\threebytwomat}[6]{\parens{\begin{array}{cc} #1 & #2 \\ #3 & #4 \\ #5 & #6 \end{array}}}
\newcommand{\fourvec}[4]{\parens{\begin{array}{c} #1 \\ #2 \\ #3 \\ #4\end{array}}}
\newcommand{\threebythreemat}[9]{\parens{\begin{array}{ccc} #1 & #2 & #3 \\  #4 & #5 & #6 \\ #7 & #8 & #9 \end{array}}}

%Sign
\DeclareMathOperator{\sgn}{sgn}
\DeclareMathOperator{\sign}{sign}

% Project specific notation
\newcommand{\betahatpost}{\betahat_{post}}
\newcommand{\betahatpostTN}{\betahat_{post}^{TN}}
\newcommand{\betahatpre}{\betahat_{pre}}
\newcommand{\betatildepost}{\betatilde_{post}}
\newcommand{\betatildepre}{\betatilde_{pre}}
\newcommand{\betapost}{\beta_{post}}
\newcommand{\gammapost}{\gamma_{post}}
\newcommand{\gammahat}{\widehat{\gamma}}
\newcommand{\gammaopt}{\gamma_{*}}
\newcommand{\gammaoptsub}[1]{\gamma_{*#1}}

\newcommand{\gammabar}{\bar{\gamma}}
\newcommand{\gammahatpost}{\widehat{\gamma}_{post}}
\newcommand{\gammahatpostTN}{\widehat{\gamma}_{post}^{TN}}
\newcommand{\gammatilde}{\ensuremath{\tilde{\gamma}}}

\newcommand{\etahat}{\widehat{\eta}}

\newcommand{\deltahat}{\widehat{\delta}}
\newcommand{\deltahatpost}{\deltahat_{post}}
\newcommand{\deltahatpre}{\deltahat_{pre}}
\newcommand{\deltapost}{\delta_{post}}
\newcommand{\deltapre}{\delta_{pre}}

\newcommand{\taupost}{\tau_{post}}
\newcommand{\taupre}{\tau_{pre}}
\newcommand{\taubarone}{\bar{\tau}_1}

\newcommand{\thetabar}{\bar{\theta}}

\newcommand{\Sigmapre}{\Sigma_{pre}}
\newcommand{\Xpre}{X_{pre}}


\newcommand{\CITNbeta}{CI^{TN}_{\beta}}
\newcommand{\CITNgamma}{CI^{TN}_{\gamma}}
\newcommand{\CITrad}{CI^{Trad}}
\newcommand{\betapre}{\beta_{pre}}
\newcommand{\ybar}{\bar{y}}
\newcommand{\xt}{x^t}

\newcommand{\boldy}{\boldsymbol{y}}
\newcommand{\boldytilde}{\boldsymbol{\tilde{y}}}
\newcommand{\boldx}{\boldsymbol{x}}
\newcommand{\boldalpha}{\boldsymbol{\alpha}}
\newcommand{\boldbeta}{\boldsymbol{\beta}}

%\newcommand{\prob}[1]{P\left( #1 \right)}

\newcommand{\tildegammaone}{\tilde{\gamma}_1}
\newcommand{\tildegammamain}{\tilde{\gamma}_{main}}

\newcommand{\taubar}{\bar{\tau}}
\newcommand{\tautilde}{\tilde{\tau}}
\newcommand{\tildetau}{\tautilde}


%\newcommand{\ubar}[1]{\underaccent{\bar}{#1}}
%\newcommand{\ubar}[1]{\underline{$#1$}}

\newcommand{\ubar}[1]{\underaccent{\bar}{#1}}

\newcommand{\Tau}{\mathrm{T}}

\newcommand{\Bcal}{\mathcal{B}}

\newcommand{\Atilde}{\tilde{A}} 
\newcommand{\Atildeone}{\tilde{A}_{(\cdot,1)}} 
\newcommand{\Atildeminus}{\tilde{A}_{(\cdot,-1)}} 
\newcommand{\Atildebone}{\Atilde_{(B,1)}}
\newcommand{\Atildeminusbone}{\Atilde_{(-B,1)}}
\newcommand{\Atildebminus}{\Atilde_{(B,-1)}}
\newcommand{\Atildeminusbminus}{\Atilde_{(-B,1)}}

\newcommand{\sigmatilde}{\tilde{\sigma}}
\newcommand{\Sigmatilde}{\tilde{\Sigma}}
\newcommand{\Ytilde}{\tilde{Y}}
\newcommand{\mutilde}{\tilde{\mu}}
\newcommand{\alphatilde}{ \tilde{\alpha} }
\newcommand{\deltatildepre}{ \tilde{\delta}_{pre} }
\newcommand{\deltatildepost}{ \tilde{\delta}_{post} }

\newcommand{\deltabar}{ \bar{\delta} }
\newcommand{\deltabarpre}{ \bar{\delta}_{pre} }
\newcommand{\Deltapre}{\Delta_{pre}}

\newcommand{\mubar}{\bar{\mu}}
\newcommand{\Yhat}{\widehat{Y}}

% Tests and CIs
\newcommand{\condtest}[1]{\psi^{C}_{#1}}
\newcommand{\condci}{\mathcal{C}^{C}_{\alpha, n}}
\newcommand{\condteststar}{\psi^{C}_{*, \alpha}}

\newcommand{\MPtest}[1]{\psi^{MP}_{#1}}
\newcommand{\LFtestcustomalpha}[1]{\psi^{LF}_{#1}}

\newcommand{\condLFtest}{\psi^{C \mhyphen LF}_{\kappa, \alpha}}
\newcommand{\condLFteststar}{\psi^{C \mhyphen LF}_{*, \kappa, \alpha}}
\newcommand{\condLFci}{\mathcal{C}^{C \mhyphen LF}_{\kappa, \alpha, n}}

\newcommand{\condFLCItest}{\psi^{C \mhyphen FLCI}_{\kappa,\alpha}}
\newcommand{\condFLCIci}{\mathcal{C}^{C \mhyphen FLCI}_{\kappa, \alpha, n}}

\newcommand{\FLCInon}[1]{\mathcal{C}^{FLCI}_{#1}}
\newcommand{\FLCI}[1]{\mathcal{C}^{FLCI}_{#1, n}}
\newcommand{\LFtest}{\psi^{LF}_{\kappa}}

\newcommand{\scaledv}{v}
\newcommand{\unscaledv}{\breve{v}}
\newcommand{\scaleda}{a}
\newcommand{\unscaleda}{\breve{a}}
\newcommand{\scaledchi}{\chi}
\newcommand{\unscaledchi}{\breve{\chi}}


\author{Jonathan Roth}
\title[Advanced DiD Mixtape Workshop]{Advanced DiD Mixtape Workshop \\ Multiple Periods and Staggered Treatment Timing}

\begin{document}

\imageframe{figures/cover_staggered.png}

\begin{frame}{Staggered Timing}
  \addtocounter{framenumber}{-1}
  \begin{wideitemize}
    \item
    Remember that in the canonical DiD model we had:

    \begin{itemize}
      \item
            Two periods and a common treatment date

      \item
            Identification from parallel trends and no anticipation

      \item
            A large number of clusters for inference
    \end{itemize}

    \item
    A very active recent literature has focused on relaxing the first assumption: \textbf{what if there are multiple periods and units adopt treatment at different times?}

    \item
    This literature typically maintains the remaining ingredients: parallel trends and many clusters

  \end{wideitemize}

\end{frame}


\begin{frame}{Overview of Staggered Timing Literature}

  \begin{enumerate}
    \item
          Negative results: TWFE OLS doesn't give us what we want with treatment effect heterogeneity

    \item
          New estimators: perform better under treatment effect heterogeneity
  \end{enumerate}


\end{frame}

\begin{frame}{Staggered timing set-up}
  \begin{wideitemize}
	\item
	Panel of observations for periods $t = 1,...,T$

    \item
    Suppose units adopt a binary treatment at different dates $G_i \in \{1,...,T \} \cup \infty$ (where $G_i = \infty$ means ``never-treated'')
    \begin{itemize}
      \item
            Literature is now starting to consider cases with continuous treatment \& treatments that turn on/off -- that lit is still developing (see Section 3.4 of review paper)
    \end{itemize}

    \item
    Potential outcomes $Y_{it}(g)$ -- depend on time and time you were first-treated

  \end{wideitemize}
\end{frame}


\begin{frame}{Extending the Identifying Assumptions}
  \begin{wideitemize}
    \item
    The key identifying assumptions from the canonical model are extended in the natural way

    \item
    \textbf{Parallel trends:} Intuitively, says that if treatment hadn't happened, all ``adoption cohorts'' would have parallel average outcomes in all periods

    $$ E[ Y_{it}(\infty) - Y_{i,t-1}(\infty) | G_i = g ] = E[ Y_{it}(\infty) - Y_{i,t-1}(\infty) | G_i = g' ] \text{ for all } g,g',t$$


    Note: can impose slightly weaker versions (e.g. only require PT post-treatment)

    \item
    \textbf{No anticipation:} Intuitively, says that treatment has no impact before it is implemented

    $$Y_{it}(g) = Y_{it}(\infty) \text{ for all } t<g$$
  \end{wideitemize}
\end{frame}


\begin{frame}{Negative results}
  \begin{wideitemize}
    \item
    Suppose we again run the regression
    \vspace{-3mm}
    $$Y_{it} = \alpha_i + \phi_t + D_{it} \beta  + \epsilon_{it}, $$
    \noindent where $D_{it} = 1[t \geq G_i]$ is a treatment indicator. 
    
    \item
    Suppose we're willing to assume no anticipation and  parallel trends across all adoption cohorts as described above

    \pause
    \vspace{-3mm}
    \item
    Good news: if treatment effects are constant across time and units, $Y_{it}(g) - Y_{it}(\infty) \equiv \tau$, then $\beta = \tau$

    \pause
    \vspace{-3mm}
    \item
    Bad news: if treatment effects are heterogeneous, then $\beta$ may put negative weights on treatment effects for some units and time periods
    \begin{itemize}
      \item
            E.g., if treatment effect depends on time since treatment, $Y_{it}(t-r) - Y_{it}(\infty) = \tau_{r}$, then some $\tau_r$s may get negative weight
    \end{itemize}
  \end{wideitemize}
\end{frame}

\begin{frame}{Where do these negative results come from?}
  \begin{wideitemize}
    \item
    The intuition for these negative results is that the TWFE OLS specification combines two sources of comparisons:

    \medskip

    \begin{enumerate}
      {\normalsize
      \item
            \textbf{Clean comparisons:} DiD's between treated and not-yet-treated units

            \medskip

      \item
            \textbf{Forbidden comparisons:} DiD's between two sets of already-treated units (who began treatment at different times)
            }
    \end{enumerate}

    \item
    These forbidden comparisons can lead to negative weights: the ``control group'' is already treated, so we run into problems if their treatment effects change over time
  \end{wideitemize}
\end{frame}

\begin{frame}{Some intuition for forbidden comparisons}
  \begin{wideitemize}
    \item
    Consider the two period model, except suppose now that our two groups are \textbf{always-treated} units (treated in both periods) and \textbf{switchers} (treated only in period 2)

    \item
    With two periods, the coefficient  $\beta$ from $ Y_{it} = \alpha_i + \phi_t + D_{it} \beta  + \epsilon_{it}$ is the same as from the first-differenced regression  $\Delta Y_i = \alpha + \Delta D_i \beta + u_i$
   
   \item
   Observe that $\Delta D_i$ is one for switchers and zero for stayers. \\ That is, the stayers are the control group! Thus,

    $$\hat\beta =  \underbrace{ \left(\bar{Y}_{Switchers, 2} - \bar{Y}_{Switchers, 1} \right) }_{\text{Change for switchers}} - \underbrace{ \left(\bar{Y}_{AT, 2} - \bar{Y}_{AT, 1} \right) }_{\text{Change for always treated}}  $$

    \item
    Problem: if the treatment effect for the always-treated grows over time, that will enter $\hat\beta$ negatively!

    \item
    With multiple periods/staggered timing, $\hat\beta$ includes both this type of comparisons and clean comparisons
  \end{wideitemize}
\end{frame}

\begin{frame}{Second Intuition for Negative Weights}
	\begin{wideitemize}
		\item
		The Frisch-Waugh-Lovell theorem says that we can obtain the coefficient $\beta$ in 	
		$Y_{it} = \alpha_i + \phi_t + D_{it} \beta  + \epsilon_{it}$ by the following two-step procedure. 
		
		\item
		First, regress the treatment indicator $D_{it}$ on the FEs (a linear probability model): $D_{it} = \tilde{\alpha}_i + \tilde{\phi}_t + \tilde{\epsilon_{it}} $
		
		\item
		
		\noindent Then run a univariate regression of $Y_{it}$ on $D_{it}-\hat{D}_{it}$ to obtain $\beta$.
		
		\begin{itemize}
			\item 
			Thus, $\beta = \frac{Cov( Y_{it}, D_{it} - \hat{D}_{it} )}{ Var(D_{it} - \hat{D}_{it})  } =  \frac{E( Y_{it} (D_{it} - \hat{D}_{it})) }{ Var(D_{it} - \hat{D}_{it})  }$
		\end{itemize}
		
		\item
		However, it's well known that the linear probability model for $D_{it}$ may have predictions outside the unit interval. If $\hat{D}_{it}>1$ even though unit $i$ is treated in period $t$, then $D_{it}- \hat{D}_{it} <0$, and thus $Y_{it}$ gets negative weight. 
	\end{wideitemize}
\end{frame}

\begin{frame}{Not just negative but weird...}
  The literature has placed a lot of emphasis on the fact that some treatment effects may get negative weights
  \begin{wideitemize}
    \item
    But even if the weights are non-negative, they might not give us the most intuitive parameter

    \item
    For example, suppose each unit $i$ has treatment effect $\tau_i$ in every period if they are treated (no dynamics). Then $\beta$ gives a weighted average of the $\tau_i$ where the weights are largest for units treated closest to the middle of the panel

    \item
    It is not obvious that these weights are relevant for policy, even if they are all non-negative!
  \end{wideitemize}
\end{frame}

\begin{frame}{Issues with dynamic TWFE}
  \begin{wideitemize}
    \item
    \citet{sun_estimating_2020} show that similar issues arise with dynamic TWFE specifications:

    \begin{equation*}
      Y_{i,t} = \alpha_i + \lambda_t +  \sum_{k \neq 0} \sun{ \gamma_k D_{i,t}^{k}} + \varepsilon_{i,t},
    \end{equation*}
    where $D_{i,t}^{k} = 1\left\{t-G_{i}=k\right\}$ are ``event-time'' dummies.

    \item
    Like for the static spec, $\sun{\gamma_k}$ may put negative weight on treatment effects after $k$ periods for some units

    \item
    SA also show that $\sun{\gamma_k}$ may be ``contaminated'' by treatment effects at lags $k' \neq k$


  \end{wideitemize}

\end{frame}


\begin{frame}{Dynamic TWFE - Continued}

  \begin{wideitemize}

    \item
    The results in SA suggest that interpreting the $\sun{\hat\gamma_k}$ for $k=1,2,...$ as estimates of the dynamic effects of treatment may be misleading


    \item
    These results also imply that pre-trends tests of the $\gamma_k$ for $k<0$ may be misleading -- could be non-zero even if parallel trends holds, since they may be ``contaminated'' by post-treatment effects!
    \pause

    \item
    The issues discussed in SA arise if dynamic path of treatment effects is heterogeneous across adoption cohorts
    \begin{itemize}
      \item
            Biases may be less severe than for ``static'' specs if dynamic patterns are similar across cohorts
    \end{itemize}


  \end{wideitemize}

\end{frame}

\begin{frame}{New estimators (and estimands!)}
  \begin{wideitemize}
    \item
    Several new (closely-related) estimators have been proposed to try to address these negative weighting issues

    \item
    The key components of all of these are:

    \begin{enumerate}
      {\normalsize
      \item
            Be precise about the target parameter (estimand) -- i.e., how do we want to aggregate treatment effects across time/units
            \bigskip

      \item
            Estimate the target parameter using only ``clean-comparisons''
            }
    \end{enumerate}
  \end{wideitemize}
\end{frame}

\begin{frame}{Example -- Callaway and Sant'Anna (2020)}
  \begin{wideitemize}
    \item
    Define $ATT(g,t)$ to be ATT in period $t$ for units first treated at period $g$,

    $$ATT(g,t) = E[ Y_{it}(g) - Y_{it}(\infty) | G_i = g] $$

    \pause
    \item
    Under PT and No Anticipation, $ATT(g,t)$ is identified as
    $$ATT(g,t) = \underbrace{E[ Y_{it} - Y_{i,g-1}| G_i = g]}_{\text{Change for cohort g}} -  \underbrace{E[ Y_{it} - Y_{i,g-1}| G_i = \infty]}_{\text{Change for never-treated units}} $$

    \item
    Why? \pause{} This is a two-group two-period comparison, so the argument is the same as in the canonical case!

  \end{wideitemize}
\end{frame}

\begin{frame}{Proof of Identification Argument}
  \begin{itemize}
    \item
          Start with
          \vspace{-3mm}
          $$E[Y_{it}- Y_{i,g-1}| G_i =g] - E[Y_{it} - Y_{i,g-1}| G_i = \infty]$$
          
          \pause
    \vspace{-3mm}
    \item
          Apply definition of POs to obtain:
          \vspace{-3mm}
          $$E[Y_{it}(g) - Y_{i,g-1}(g) | G_i =g] - E[Y_{ig}(\infty) - Y_{i,g-1}(\infty) | G_i = \infty]$$
          
          \pause
    \vspace{-3mm}
    \item
          Use No Anticipation to substitute $Y_{i,g-1}(\infty)$ for $Y_{i,g-1}(g)$:
          \vspace{-3mm}
          $$E[Y_{it}(g) - Y_{i,g-1}(\infty) | G_i =g] - E[Y_{ig}(\infty) - Y_{i,g-1}(\infty) | G_i = \infty]$$
          
          
          \pause
    \vspace{-3mm}
    \item
          Add and subtract $E[ Y_{it}(\infty) | G_i =g] $ to obtain:
          \begin{align*}
              & \electricViolet{E[ Y_{it}(g) - Y_{it}(\infty) | G_i =g]} +                                                                                           \\
              & \hspace{1cm} \sun{\left[ E[Y_{it}(\infty) - Y_{i,g-1}(\infty) | G_i =g] - E[Y_{ig}(\infty) - Y_{i,g-1}(\infty) | G_i = \infty] \right]}
          \end{align*}

          \pause
          \vspace{-3mm}
    \item
          Cancel the \sun{last term} using PT to get $\electricViolet{E[Y_{it}(g) - Y_{it}(\infty) | G_i = g] = ATT(g,t)}$
  \end{itemize}


\end{frame}
\begin{frame}{Example -- Callaway and Sant'Anna (2020)}
  \begin{wideitemize}
    \item
    Define $ATT(g,t)$ to be ATT in period $t$ for units first treated at period $g$,

    $$ATT(g,t) = E[ Y_{it}(g) - Y_{it}(\infty) | G_i = g] $$
    
    \pause
    \vspace{-3mm}
    \item
    Under PT and No Anticipation,
    $$ATT(g,t) = \underbrace{E[ Y_{it} - Y_{i,g-1}| G_i = g]}_{\text{Change for cohort g}} -  \underbrace{E[ Y_{it} - Y_{i,g-1}| G_i = \infty]}_{\text{Change for never-treated}} $$
    
    \pause
    \vspace{-3mm}
    \item
    We can then estimate this with sample analogs:
    $$\widehat{ATT}(g,t)= \underbrace{\widehat{E}[ Y_{it} - Y_{i,g-1} | G_i = g]}_{\text{Sample change for cohort g}} -  \underbrace{\widehat{E}[ Y_{it} - Y_{i,g-1} | G_i = \infty]}_{\text{Sample change for never-treated}} $$
    
    \vspace{-3mm}
    where $\hat{E}$ denotes sample means.

  \end{wideitemize}
\end{frame}

\begin{frame}{Aggregation schemes}
  \begin{wideitemize}
    \item
    If have a large number of observations and relatively few groups/periods, can report $\widehat{ATT}(g,t)$'s directly.

    \item
    If there are many groups/periods, the $\widehat{ATT}(g,t)$ may be very imprecisely estimated and/or too numerous to report concisely

  \end{wideitemize}
\end{frame}

\begin{frame}{Aggregation schemes}
  \begin{wideitemize}
    \item
    In these cases, it is often desirable to report sensible averages of the $\widehat{ATT}(g,t)$'s.

    \pause
    \item
    One of the most useful is to report event-study parameters which aggregate $\widehat{ATT}(g,t)$'s at a particular lag since treatment
    \begin{itemize}
      \item E.g. $\hat{\theta}_k = \sum_g \widehat{ATT}(g, t+k)$ aggregates effects for cohorts in the $k$th period after treatment

      \item Can also construct for $k<0$ to estimate ``pre-trends''
    \end{itemize}

    \pause
    \item
    C\&S discuss other sensible aggregations too -- e.g., if interested in whether treatment effects differ across good/bad economies, may want to ``calendar averages'' that pool the $\widehat{ATT}(t,g)$ for the same year
  \end{wideitemize}

\end{frame}

\begin{frame}{Comparisons of new estimators}
  \begin{wideitemize}
    \item
    Callaway and Sant'Anna also propose an analogous estimator using \textit{not-yet-treated} rather than never-treated units.

    \vspace{-3mm}
    \item
    \citet{sun_estimating_2020} propose a similar estimator but with different comparisons groups (e.g. using last-to-be treated rather than not-yet-treated)

    \vspace{-3mm}
    \item
    \citet{borusyak_revisiting_2021}, \citet{Wooldridge2021a}, \citet{gardner_two-stage_2021} propose ``imputation'' estimators that estimate the counterfactual $\hat{Y}_{it}(0)$ using a TWFE model that is fit using only pre-treatment data
    \begin{itemize}
      \item Main difference from C\&S is that this uses more pre-treatment periods, not just period $g-1$

      \item This can sometimes be more efficient (if outcome not too serially correlated), but also relies on a stronger PT assumption that may be more susceptible to bias
    \end{itemize}

    \vspace{-3mm}
    \item \citet{roth_efficient_2021} show that you can get even more precise estimates if you're willing to assume treatment timing is ``as good as random''

  \end{wideitemize}
\end{frame}

\begin{frame}{Personal advice}
  \begin{wideitemize}
    \item
    Don't freak out about this new literature!

    \pause
    \item
    In most cases, using the ``new'' DiD methods will not lead to a big change in your results (empirically, TE heterogeneity is not \textit{that} large in most cases)
    \begin{itemize}
      \item
            The exceptions are cases where there are periods where almost all units are treated -- this is when ``forbidden comparisons'' get the most weight
    \end{itemize}

    \pause
    \item
    The most important thing is to be precise about who you want the comparison group to be and to choose a method that only uses these ``clean comparisons''

    \pause
    \item
    In my experience, the difference between the new estimators is typically not that large -- can report multiple new methods for robustness (to make your referees happy!)
  \end{wideitemize}
\end{frame}


\backupbegin
\begin{frame}[allowframebreaks,noframenumbering,plain]{References}
  \bibliography{Bibliography.bib}
\end{frame}
\backupend
\end{document}
